
\vspace*{0.1in}
\noindent\textbf{4.1. Projeden Elde Edilmesi Öngörülen Çıktılara İlişkin Bilgiler}
% Bu bölümde, projeden elde edilmesi öngörülen çıktılara yer verilmelidir. Söz konusu çıktılar, amaçlarına göre belirlenen kategorilere ayrılarak belirtilmeli, nicel gösterge ve hedeflere dayandırılmalı ve varsa bu çıktıları kullanacak kurum/kuruluş(lar)a ilişkin bilgi verilmelidir. Her bir çıktının elde edilmesinin öngörüldüğü zaman aralığı belirtilmelidir.

\vspace*{0.1in}
\begin{tabular}{|
>{\columncolor[HTML]{C0C0C0}}l |l|l|}
\hline
\multicolumn{1}{|c|}{\cellcolor[HTML]{C0C0C0}\textbf{Çıktı Türü}} & \multicolumn{1}{c|}{\textbf{Öngörülen Çıktı (lar)}} & \multicolumn{1}{c|}{\textbf{\begin{tabular}[c]{@{}c@{}}Çıktının Elde Edilmesi için\\ Öngörülen Zaman Aralığı (*)\end{tabular}}} \\ \hline
\begin{tabular}[c]{@{}l@{}}\textbf{Bilimsel/Akademik Çıktılar} (Bildiri, \\ Makale, Kitap Bölümü, Kitap vb.):\end{tabular} &  &  \\ \hline
\begin{tabular}[c]{@{}l@{}}\textbf{Ekonomik/Ticari/Sosyal Çıktılar}\\ (Ürün, Prototip, Patent, Faydalı Model, \\ Üretim İzni, Tescil, Görsel/İşitsel Arşiv, \\ Envanter/Veri Tabanı/Belgeleme Üretimi, \\ Spin-off/Start- up Şirket vb.):\end{tabular} &  &  \\ \hline
\begin{tabular}[c]{@{}l@{}}\textbf{Araştırmacı Yetiştirilmesi ve Yeni} \\ \textbf{Proje(ler) Oluşturulmasına Yönelik} \\ \textbf{Çıktılar} (Yüksek Lisans/Doktora/Tıpta \\ Uzmanlık/Sanatta Yeterlik Tezleri ve \\ Ulusal/Uluslararası Yeni Proje vb.)\end{tabular} &  &  \\ \hline
\end{tabular} \\
{\footnotesize (*) Proje başlangıcından itibaren 6 aylık süreler halinde belirtilmelidir (Örn. 0-6 ay/6-12 ay/12-18 ay, Proje sonrası vb.). }

\vspace*{0.1in}
\noindent\textbf{4.2. Proje Çıktılarının Paylaşımı ve Yayılımı}
% Proje faaliyetleri boyunca elde edilecek çıktıların ve ulaşılacak sonuçların ilgili paydaşlar ve olası kullanıcılara ulaştırılması ve yayılmasına yönelik yapılacak olan toplantı, çalıştay, eğitim, web sitesi, medya, fuar, proje pazarı ve benzeri etkinlikler aşağıdaki tabloda verilmelidir. 

\begin{center}
\textbf{PROJE ÇIKTILARININ PAYLAŞIMI VE YAYILIMI TABLOSU (*)}
\end{center}

\begin{tabular}{|
>{\columncolor[HTML]{FFFFFF}}l |l|l|}
\hline
\multicolumn{1}{|c|}{\cellcolor[HTML]{C0C0C0}\begin{tabular}[c]{@{}c@{}}\textbf{Etkinlik Türü} (Toplantı, Çalıştay, Eğitim, \\ Web Sayfası, Görsel/Yazılı/Sosyal \\ Medya, Fuar, Proje Pazarı vb.)\end{tabular}} & \multicolumn{1}{c|}{\textbf{Paydaş / Olası Kullanıcılar}} & \multicolumn{1}{c|}{\textbf{Etkinliğin Zamanı ve Süresi}} \\ \hline
 &  &  \\ \hline
 &  &  \\ \hline
 &  &  \\ \hline
\end{tabular}\\ %
{\footnotesize (*) Tablodaki satırlar gerektiği kadar genişletilebilir ve çoğaltılabilir.}

\vspace*{0.1in}
\noindent\textbf{4.3. Projeden Oluşması Öngörülen Etkilere İlişkin Bilgiler}
% Proje başarıyla gerçekleştirildiği takdirde projeden oluşması öngörülen 
%    • Toplumsal/kültürel etki,
%    • Ekonomik etki, 
%    • Ulusal Güvenlik etkisi 

%Proje Başvuru Sistemi (PBS)'nde seçilen 11. Kalkınma Planı hedefleri ve politikaları çerçevesinde hedef kitle/alan belirtilerek açıklanmalıdır. Beklenen etkiler doğrulanabilir ve ölçülebilir olmalıdır. Etkilerin oluşma zamanına ilişkin öngörüler belirtilmelidir.

\vspace*{0.1in}
\begin{tabular}{|
>{\columncolor[HTML]{C0C0C0}}l |l|l|}
\hline
\multicolumn{1}{|c|}{\cellcolor[HTML]{C0C0C0}\textbf{Etki Türü}} & \multicolumn{1}{c|}{\textbf{\begin{tabular}[c]{@{}c@{}}Öngörülen Etki Türü ve \\ Kalkınma Planıyla İlişkisi\end{tabular}}} & \multicolumn{1}{c|}{\textbf{\begin{tabular}[c]{@{}c@{}}Etkinin Oluşması\\ Öngörülen Zaman (*)\end{tabular}}} \\ \hline
\begin{tabular}[c]{@{}l@{}}
\textbf{Toplumsal/Kültürel Etki:}\\ \\ 
• Yaşam Kalitesine Katkı,\\ 
• Sürdürülebilir Çevre ve Enerjiye Katkı,\\ 
• Refah veya Eğitim Seviyesinin İyileştirilmesine \\    Katkı,  \\ 
• Ülke ya da Dünya Düzeyinde Önemli Bir\\    Sosyal Soruna Getirilecek Çözümler vb.\\ 
• Proje Sonuçlarını Uygulayan Kurum/\\   Kuruluş
\end{tabular} &  &  \\ \hline
\begin{tabular}[c]{@{}l@{}}
\textbf{Ekonomik Etki:}\\     
• Potansiyel Sektörel Uygulama Alanları,\\     
• Küresel Pazar Öngörüleri,\\     
• İstihdam Katkısı,\\     
• Rekabetçilik (İhracata Etkisi, İthal \\        İkamesi, Yabancı Sermaye Yatırımının \\        Tetiklenmesi vb.)
\end{tabular} &  &  \\ \hline
\begin{tabular}[c]{@{}l@{}}
\textbf{Ulusal Güvenlik Etkisi:}\\ 
• Siber güvenlik, \\     
• Enerji güvenliği, \\     
• Sınır güvenliği, \\     
• Gıda güvenliği,\\     
• Ekonomik güvenlik vb.
\end{tabular} &  &  \\ \hline
\end{tabular} \\ %
{\footnotesize (*) Proje başlangıcından itibaren 6 aylık süreler halinde belirtilmelidir (Örn. 0-6 ay/6-12 ay/12-18 ay, Proje sonrası vb.)}