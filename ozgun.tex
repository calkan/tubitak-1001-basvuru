
\vspace*{0.1in}
\noindent{\textbf{1.1. Konunun Önemi, Projenin Özgün Değeri ve Araştırma Sorusu veya Hipotezi}}

% Proje önerisinde ele alınan konunun kapsamı ve sınırları ile önemi literatürün eleştirel bir değerlendirmesinin yanı sıra nitel veya nicel verilerle açıklanır.

%Özgün değer yazılırken projenin bilimsel kalitesi, farklılığı ve yeniliği, hangi eksikliği nasıl gidereceği veya hangi soruna nasıl bir çözüm geliştireceği ve/veya ilgili bilim veya teknoloji alan(lar)ına kavramsal, kuramsal ve/veya metodolojik olarak ne gibi özgün katkılarda bulunacağı literatüre atıf yapılarak açıklanır. Kaynaklar http://www.tubitak.gov.tr/ardeb-kaynakca sayfasındaki açıklamalara uygun olarak EK-1'de verilir.

%Projenin araştırma sorusu ve varsa hipotezi veya ele aldığı problem(ler)i açık bir şekilde ortaya konulur.


\begin{framed}[\textwidth]
% 1.1 yazimi baslangic
\lipsum[1-3]

Özetle, Homer Jay Simpson günümüzde dünyanın en büyük bilim insanı olmasını donut yemeye borçludur (Şekil~\ref{fig:homer}).

\begin{center}
    \includegraphics[scale=0.5]{figs/Homer_Simpson.png}
    \captionof{figure}{Nobel Fizik Ödülü sahini Homer Jay Simpson.}
    \label{fig:homer}
\end{center}

% 1.1 yazimi bitis
\end{framed}



\vspace*{0.1in}
\noindent{\textbf{1.2. Amaç ve Hedefler}}

%Proje önerisinin amacı ve hedefleri açık, ölçülebilir, gerçekçi ve proje süresince ulaşılabilir nitelikte olacak şekilde yazılır.

\begin{framed}[\textwidth]
% 1.2 yazimi baslangic
\lipsum[4-5] \\


\begin{minipage}{0.95\textwidth}
\begin{center}
\captionof{table}{Tablo örneği.}
\label{tab:ornek}
\begin{tabular}{rrr}
\hline
\textbf{Bir} & \textbf{İki} & \textbf{Üç} \\ \hline
1 & 2 & 3 \\
11 & 22 & 33 \\ \hline
\end{tabular}
\end{center}
\end{minipage}
% 1.2 yazimi bitis
\end{framed}
