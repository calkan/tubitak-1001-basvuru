
\vspace*{0.1in}
\noindent{\textbf{1.1. Konunun Önemi, Projenin Özgün Değeri ve Araştırma Sorusu veya Hipotezi}}

% Proje önerisinde ele alınan konunun kapsamı ve sınırları ile önemi literatürün eleştirel bir değerlendirmesinin yanı sıra nitel veya nicel verilerle açıklanır.

%Özgün değer yazılırken projenin bilimsel kalitesi, farklılığı ve yeniliği, hangi eksikliği nasıl gidereceği veya hangi soruna nasıl bir çözüm geliştireceği ve/veya ilgili bilim veya teknoloji alan(lar)ına kavramsal, kuramsal ve/veya metodolojik olarak ne gibi özgün katkılarda bulunacağı literatüre atıf yapılarak açıklanır. Kaynaklar http://www.tubitak.gov.tr/ardeb-kaynakca sayfasındaki açıklamalara uygun olarak EK-1'de verilir.

%Projenin araştırma sorusu ve varsa hipotezi veya ele aldığı problem(ler)i açık bir şekilde ortaya konulur.


\begin{center}
\fbox{ %
\parbox{\textwidth}
{
% 1.1 yazimi baslangic
\lipsum[1-3]
% 1.1 yazimi bitis
} %
}
\end{center}



\vspace*{0.1in}
\noindent{\textbf{1.2. Amaç ve Hedefler}}

%Proje önerisinin amacı ve hedefleri açık, ölçülebilir, gerçekçi ve proje süresince ulaşılabilir nitelikte olacak şekilde yazılır.

\begin{center}
\fbox{ %
\parbox{\textwidth}
{
% 1.2 yazimi baslangic
\lipsum[4-5]
% 1.2 yazimi bitis
} %
}
\end{center}
